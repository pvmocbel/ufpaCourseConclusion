%Nesse trabalho foi desenvolvido um software de modelagem 3D que permite construir ambientes virtuais baseados nas características dos reais. Ele se utiliza das técnicas de modelagem mais usadas no mercado, além de também utilizar da teoria de grafo de cena. Possibilitando que o cenário nele desenvolvido simule a propagação de ondas eletromagnéticas usando a método FDTD.\\

%Um estudo de caso foi realisado, obtendo um resultado satisfatório que comprovou que tanto a modelagem quanto a conexão com o simulador LANE-SAGS funcionaram de forma desejada. Assim, mostrando que a interface desenvolvida cumpriu com os requisitos prospostos para esse projeto.

%\subsection{Trabalhos Futuros}
%Como prosseguimento natural desse trabalho, pretende-se:
%\begin{itemize}
%	\item{Melhorar a classe de conexão irrlicht-qt}
%	\item{Adicionar novos objetos ao conjunto já existentes.}
%	\item{Possibilitar o agrupamento e desagrupamento de objetos no ambiente virtual.}
%	\item{Possibilitar a imersão nos cenários criados através de uma re-estruturação do software.}

%\end{itemize}

Ainda estou terminando.
