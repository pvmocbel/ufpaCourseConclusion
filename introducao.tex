	Com o passar dos anos, a computação tem evoluído de forma exponencial\cite{comp_history}, permitindo cada vez mais ao ser humano: armazenar grande quantidades de dados, simular ambientes de grandes magnitudes, fazer previsão de eventos, se comunicar audiovisualmente, etc. Todo esse avanço vem refletindo de forma crescente na vida do homem.

	Mais especificamente na área de telecomunicações, pode-se dizer que esse crescimento vem possibilitando avanços nos estudos e aplicações relacionadas às simulações de ondas eletromagnéticas e antenas. O método das Diferenças Finitas no Domínio do Tempo (FDTD)\cite{fdtd_intro} é um dos mais usados (e antigos) nos estudos relativos à propagações de ondas eletromagnéticas em ambientes \textit{indoor} e \textit{outdoor}\cite{rodrigo_intro}.

	Porém, muitas vezes, a representação virtual de desses ambientes não é uma tarefa fácil. Dessa forma, surge a necessidade de criação de \textit{softwares} que auxiliem nessa construção. Eles podem ser de modelagem 2D ou 3D desde de que permitam a criação de estruturas básicas, como: triângulos, círculos, cubos, esferas e pirâmides; como também outras bem mais complexas(fractais, estruturas periódicas e prédios), gerando sempre a base que contenha as coordenadas de cada objeto desse cenário.

\section{Objetivos}
	Esse trabalho tem com finalidade a construção de um \textit{software} que permita modelar ambientes tridimensionais (usando conceitos de realidade virtual) que se aproximem, da melhor forma possível, um cenário real. A partir desse universo virtual, objetiva-se obter a malha compatível com o programa LANE-SAGS. Através dele, simular a propagação de ondas eletromagnéticas utilizando o método FDTD. Assim possibilitando analisar tanto aterramento e descargas elétricas quanto a resposta desse cenário a propagação do sinal de uma antena.

	Dessa forma, os principais objetivos desse trabalho são:
\begin{enumerate}
\item {Representar estruturas tridimensionais por meio de alguns objetos básicos, como: cubo, esfera, cilindro e cone.}
\item {Criar AV com base em ambientes reais. }
\item {Gerar uma base de dados com as característica de cada estrutura, sua posição, dimensão e parâmetros físicos($\mu$, $\sigma$ e $\epsilon$).}
\item {Gerar malha compatível com o simulador LANE-SAGS, efetuar a simulação e comparar os resultados com experimentos reais relativos à propagação eletromagnética em um ambiente \textit{indoor}.}
\end{enumerate}

\section{Organização do Trabalho}
	Este trabalho foi estruturado da seguinte forma:
\begin{itemize}
\item \textbf{Capítulo 2:} Trata da base teórica necessária para realização do trabalho, falando de computação gráfica, técnicas de modelagem de sólidos, teoria dos grafos de cena e o método FDTD.
\item \textbf{Capítulo 3:} Fala sobre a interface desenvolvida nesse trabalho, suas aplicações, classes, aparência, objetos básicos, manipuladores de cena, região de análise e seus parâmetros, geração de malha, posicionamento de câmera, atalhos de teclado, eventos de mouse, etc. 
\item \textbf{Capítulo 4:} Mostra uma aplicação usando o \textit{software} com seus resultados (comparados com os obtidos via medição), seguindo todos os passos desde da criação de um ambiente até sua simulação no LANE-SAGS.
\end{itemize}
