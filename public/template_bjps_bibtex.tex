\documentclass[12pt,twoside,reqno,final]{bjps}

\def\publname{\textit{Brazilian Journal of Probability and Statistics}} 
\issueinfo{32}{1}{April}{2007}
\pagespan{1}{5} 
\def\ISSN{0103-0752}
\PII{ISSN \ISSN}
\copyrightinfo{1987}{Brazilian Statistical Association}

\usepackage{amsfonts,amssymb,latexsym,texnames}
\usepackage{graphicx}

%% use txfonts (Times) 
\usepackage{txfonts}

\usepackage{hyperref,natbib}

% Optional
%\usepackage[active]{srcltx}

\usepackage[letterpaper,left=1.40in,right=1.40in,top=1.25in,bottom=1.25in]{geometry}


\theoremstyle{plain}
\numberwithin{equation}{section}

\let\mathbb=\varmathbb
\let\Bbbk=\varBbbk
\def\E{{\mathbb E}}
\def\R{{\mathbb R}}

\newtheorem{theorem}{Theorem}
\newtheorem{corollary}{Corollary}
\newtheorem{lemma}{Lemma}
\newtheorem{proposition}{Proposition}

%\setlength{\textwidth}{6truein}


\begin{document}

\title[Template for the BJPS]
      {\LaTeX\ Template for Preparation of Manuscripts for the Brazilian Journal of Probability and Statistics}
 
\author[F.\ Cribari-Neto]{Francisco Cribari--Neto}
\address{Departamento de Estat\'{\i}stica\\
Universidade Federal de Pernambuco\\
Cidade Universit\'{a}ria\\
Recife/PE, 50740--540, Brazil}
\email[F.\ Cribari--Neto]{cribari@de.ufpe.br}
\urladdr[F.\ Cribari--Neto]{\url{http://www.de.ufpe.br/~cribari}}

\author[A.C.\ Frery]{Alejandro C.\ Frery}
\address{Instituto de Computa\c c\~ao\\
Universidade Federal de Alagoas\\
BR 104 Norte km 97\\
Macei\'o/AL, 57072--970, Brazil}
\email[A.C.\ Frery]{acfrery@pesquisador.cnpq.br}
\urladdr[A.C.\ Frery]{\url{http://www.tci.ufal.br/frery}}

\thanks{Received xx/xx/xx, accepted yy/yy/yy}

\keywords{Template, \LaTeX, scientific typesetting}

\begin{abstract}
This document is intended to serve as a template for authors willing to type their manuscripts in \LaTeX\ for the Brazilian Journal of Probability and Statistics.
Examples of the some of the most important features are included.
The style based on the widespread \verb+amsart+ class, while the bibliographic style is a modification of the ``author-year'' Style Manual for Authors, Editors and Printers of Australian Government Publications (AGSM).
Along with this template file, we provide a 
\BibTeX\ example file and a file which contains a PostScript figure.
\end{abstract}

\maketitle

\section{Introduction}\label{sec:intro}

The files that compose this style template are the following:
\begin{enumerate}
\item\label{tempbib} \verb+template_bjps_bibtex.tex+: The template file for \BibTeX\ users.
\item\label{bib} \verb+examples.bib+: A \BibTeX\ file with examples used in the compilation of the first file.
\item\label{bibstyle} \verb+agsm_url.bst+: The \BibTeX\ style.
\item\label{tempnobib} \verb+template_bjps_nobibtex.tex+: The template file for users that intend to type the references directly into the source file.
\item\label{class} \verb+bjps.cls+: The journal style class, which is based on the \verb+amsart+ class (from the American Mathematical Society). 
\item\label{fig} \verb+furado.ps+: The PostScript file containing the figure used in the document. 
\end{enumerate}
Users that will use \BibTeX should copy files numbers~\ref{tempbib}, \ref{bib}, \ref{bibstyle}, and~\ref{fig} to their working directory.
Users that will not use \BibTeX need only files numbers~\ref{tempnobib} and~\ref{fig}.


An introduction should provide a nice bibliographic review.
Such review may consist of books, either printed~\citep{MaronnaMartinYohai:book:2006} or published online~\citep{manski_analog}.
A style for edited books is also provided, as can be seen in the work by~\citet{Matthews01}.

Usually, the main references are from research journals.
These can be cited, as already presented, between parentheses \citep{AllendeFreryetal:JSCS:05} or integrated into the text as is done in the work by \citet{Nakamuraetal:ACMCSUR}.

Papers in conference proceedings \citep[as, e.g., the work by][]{FreryFerrero:Sibgrapi:2006} can also be cited.
Notice how comments can be included using brackets in the \verb+\citep+ command.

References are sorted in alphabetical order following the name of the first author.
We recommend that the titles of journals should not be abbreviated.
When used, however, abbreviations should conform with the Index of Mathematical Papers.

The use of \verb+bibtex+ is highly recomended (for more information, details and software, see \url{http://www.ctan.org}).
Those authors that prefer to type their references directly in the manuscript, will find commented examples in the source file.

Along with this file (\verb+template_bjps_bibtex.tex+), we provide the \BibTeX\ example file \verb+examples.bib+ which contains a diversity of entries.

The compilation of the current template source file (\verb+template_bjps_bibtex.tex+) together with the example \verb+bibtex+ file 
(\verb+examples.bib+) was carried out a \texttt{Linux} system as follows:
\begin{verbatim}
author@jsbach$ latex template_bjps_bibtex
author@jsbach$ bibtex template_bjps_bibtex
author@jsbach$ latex template_bjps_bibtex
author@jsbach$ latex template_bjps_bibtex
\end{verbatim}

Users of Windows, Unix and other operational systems should refer to each platform-specific documentation.

\section{Equations, Figures and Tables}

Only equations that are referenced in the text should be numbered following the ``(section.number)'' convention, and the numbering should appear flushed to the right.
Consider, for instance, the standard Gaussian distribution. The corresponding density is
$$
f(x) = \frac1{\sqrt{2\pi}} \exp\{-{x^2}/{2} \},
$$
with $x \in \R$. The standard exponential density is
\begin{equation}
g(x) = \exp\{-x\},
\label{dens:exp}
\end{equation}
with $x \in \R_+$.
The density given in equation~\eqref{dens:exp} conveniently scales by any positive number.

Tables are quite conveniently handled by \LaTeX.
All of them should be sequentially numbered.
Consider, for instance, Table~\ref{Ta:beta1erroexp}, and notice that its caption is above the table.

\begin{table}[hbt]
\caption{{Confidence intervals for $\beta_1$: 
coverages (\%) and lengths; unbalanced design and skewed errors.}}\label{Ta:beta1erroexp}
\begin{tabular}{|c|c|c|c|c|c|c|c|}\hline
& & \multicolumn{2}{|c|}{$n=20$}& \multicolumn{2}{|c|}{$n=60$}&
\multicolumn{2}{|c|}{$n=100$} \\\hline & interval&cov.\ &length
&cov.\ &length &cov.\ &length \\\hline
 &HC0&76.87&0.25&87.31&0.20&89.85&0.17\\\cline{2-8}
 &HC2&87.26&0.37&89.42&0.22&90.94&0.18\\\cline{2-8}
 $\lambda=1$ &HC3&94.47&0.66&91.27&0.25&92.23&0.20\\\cline{2-8}
 &HC4&98.53&2.75&93.52&0.31&93.59&0.22\\\cline{2-8}
 &OLS&94.37&0.44&94.67&0.25&95.06&0.19\\\hline \hline
 &HC0&44.19&0.31&74.52&0.41&81.71&0.36\\\cline{2-8}
 &HC2&64.35&0.51&78.22&0.47&83.67&0.39\\\cline{2-8}
 $\lambda \approx 9$&HC3&83.09&0.97&81.69&0.54&85.43&0.43\\\cline{2-8}
 &HC4&95.93&4.21&87.37&0.71&88.43&0.50\\\cline{2-8}
 &OLS&71.95&0.49&68.97&0.29&68.13&0.23\\\hline \hline
 &HC0&25.72&0.45&70.26&0.80&79.74&0.72\\\cline{2-8}
 &HC2&54.39&0.81&74.57&0.92&82.06&0.78\\\cline{2-8}
 $\lambda \approx 49$&HC3&83.54&1.63&78.68&1.06&84.00&0.85\\\cline{2-8}
&HC4&96.85&7.26&85.11&1.43&87.53&1.00\\\cline{2-8}
&OLS&40.62&0.57&45.14&0.38&46.79&0.30\\\hline
\end{tabular}
\end{table}

Large tables can be rotated, i.e., displayed in landscape mode 
(horizontally) 
using \verb+\begin{sidewaystable}[p]+ to mark its beginning and 
\verb+\end{sidewaystable}+ to mark its end. 
For that, make sure that the \verb+rotating+ package is loaded. 

Several graphic formats are handled by the \verb+graphicx+ package, but one of the most convenients is PostScript (PS).
This package offers a wide variety of options for resizing, cropping and rotating figures.
Figure~\ref{fig:phantoms} presents an image stored in this format, and inserted in the manuscript with the \verb+\includegraphics+ command.
Notice that captions are below the graphical element.

\begin{figure}[hbt]
\centering
\includegraphics[width=.48\linewidth]{furado.ps}
\caption{Observed sphere with holes: amplitude image at the focal plane.}\label{fig:phantoms}
\end{figure}	

The \verb+rotating+ package can be used to display figures in landscape 
mode.

\section{Other \LaTeX\ features}

\LaTeX\ is a very rich typesetting system, and authors are encouraged to use its features as, for instances, lists of assumptions:
\begin{description}
\item[A1] The model $y=X\beta + \varepsilon$ is correctly specified; 
\item[A2] $\E(\varepsilon_i)=0,$ $i=1,\ldots,n$;
\item[A3] $\E(\varepsilon_i ^2)= {\rm var}(\varepsilon_i) = \sigma_i^2$
($0 < \sigma_i^2 < \infty$), $i=1,\ldots,n$;
\item[A3'] ${\rm var}(\varepsilon_i) = \sigma^2$, $i=1,\ldots,n$
($0< \sigma^2 < \infty$);
\item[A4] $\E(\varepsilon_i \varepsilon_j)=0$ $\forall \, i\neq j$;
\item[A5] $\lim_{n\rightarrow\infty} n^{-1}(X'X) = Q$, where 
$Q$ is a positive definite matrix. 
\end{description}

\begin{theorem}[Usual environments are provided]\label{theo:1}
Environments for theorems, colloraries, lemmas and propositions are provided in the style.
\end{theorem}

\begin{corollary}There is an environment for corollaries
\end{corollary}

\begin{proof}
The proof follows from Theorem~\ref{theo:1}.
\end{proof}

Theorems and corollaries should also be sequentially numbered, regardless the section they appear.

\section*{Appendix}

Place proofs and thechnical details in an appendix. 

\section*{Acknowledgements}

Research supported by CNPq. We also thank Johann Sebastian Bach for inspiration.

%% References can be managed by BibTeX, as shown in the following code

\small

\bibliographystyle{agsm_url}
\bibliography{examples}

\end{document}
