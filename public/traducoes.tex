%%%%%%%%%%%%%%%%%%%%%%%%%%%%%%%%%%%%%%%%%%%%%%%%%%
%% Snip para criacao de comando
%%	\traducoes{Texto em portugues}{Texto em ingles}
%%
%% Esse comando serve para selecionar um dos blocos, dependendo da
%% saida do comando de controle \selecionaPortugues
%%
%% O usuario deve fornecer o comando \selecionaPortugues, com argumento
%% 'true' ou 'false'. O default, 'true', selecionará o primeiro bloco.
%%
%% Exemplo de uso:
%%	\newcommand{\selecionaPortugues}{true}
%%	%%%%%%%%%%%%%%%%%%%%%%%%%%%%%%%%%%%%%%%%%%%%%%%%%%
%% Snip para criacao de comando
%%	\traducoes{Texto em portugues}{Texto em ingles}
%%
%% Esse comando serve para selecionar um dos blocos, dependendo da
%% saida do comando de controle \selecionaPortugues
%%
%% O usuario deve fornecer o comando \selecionaPortugues, com argumento
%% 'true' ou 'false'. O default, 'true', selecionará o primeiro bloco.
%%
%% Exemplo de uso:
%%	\newcommand{\selecionaPortugues}{true}
%%	%%%%%%%%%%%%%%%%%%%%%%%%%%%%%%%%%%%%%%%%%%%%%%%%%%
%% Snip para criacao de comando
%%	\traducoes{Texto em portugues}{Texto em ingles}
%%
%% Esse comando serve para selecionar um dos blocos, dependendo da
%% saida do comando de controle \selecionaPortugues
%%
%% O usuario deve fornecer o comando \selecionaPortugues, com argumento
%% 'true' ou 'false'. O default, 'true', selecionará o primeiro bloco.
%%
%% Exemplo de uso:
%%	\newcommand{\selecionaPortugues}{true}
%%	%%%%%%%%%%%%%%%%%%%%%%%%%%%%%%%%%%%%%%%%%%%%%%%%%%
%% Snip para criacao de comando
%%	\traducoes{Texto em portugues}{Texto em ingles}
%%
%% Esse comando serve para selecionar um dos blocos, dependendo da
%% saida do comando de controle \selecionaPortugues
%%
%% O usuario deve fornecer o comando \selecionaPortugues, com argumento
%% 'true' ou 'false'. O default, 'true', selecionará o primeiro bloco.
%%
%% Exemplo de uso:
%%	\newcommand{\selecionaPortugues}{true}
%%	\input{traducoes}
%%	... dentro do documento ...
%%	\traducoes{Ola, mundo!}{Hello, world!}
%%
%% Ultima revisao: Igor Almeida Qua Nov 24 14:09:44 BRT 2010

\usepackage{ifthen}
\providecommand{\selecionaPortugues}{true}
\newcommand{\traducoes}[2]{%
	\ifthenelse{\equal{\selecionaPortugues}{true}}{%
		#1
	}{%
		#2
	}%
}


%%	... dentro do documento ...
%%	\traducoes{Ola, mundo!}{Hello, world!}
%%
%% Ultima revisao: Igor Almeida Qua Nov 24 14:09:44 BRT 2010

\usepackage{ifthen}
\providecommand{\selecionaPortugues}{true}
\newcommand{\traducoes}[2]{%
	\ifthenelse{\equal{\selecionaPortugues}{true}}{%
		#1
	}{%
		#2
	}%
}


%%	... dentro do documento ...
%%	\traducoes{Ola, mundo!}{Hello, world!}
%%
%% Ultima revisao: Igor Almeida Qua Nov 24 14:09:44 BRT 2010

\usepackage{ifthen}
\providecommand{\selecionaPortugues}{true}
\newcommand{\traducoes}[2]{%
	\ifthenelse{\equal{\selecionaPortugues}{true}}{%
		#1
	}{%
		#2
	}%
}


%%	... dentro do documento ...
%%	\traducoes{Ola, mundo!}{Hello, world!}
%%
%% Ultima revisao: Igor Almeida Qua Nov 24 14:09:44 BRT 2010

\usepackage{ifthen}
\providecommand{\selecionaPortugues}{true}
\newcommand{\traducoes}[2]{%
	\ifthenelse{\equal{\selecionaPortugues}{true}}{%
		#1
	}{%
		#2
	}%
}

